\section{Обзор существующих решений в области калибровки рекомендательных систем}
\subsection{Обзор существующих сервисов и инструментов}
Рекомендательные системы все чаще используются в различных сервисах. 
Рассмотрим крупнейшие сервисы, внедрившие рекомендательные
системы.

Наиболее часто рекомендации применяются для рекомендаций фильмов. 
\textbf{Netflix} -- одна из передовых компаний в области рекомендаций контента,
в которой на данный момент применятся огромное количество подходов и методов
для рекомендаций не только фильмов и сериалов, но и постеров, а иногда
для порядка серий.


  Также, в последнее время, многи социальные сети начинают внедрять
  рекомендательные алгоритмы в свои сервисы. Так, например \textbf{Instagram}
  применяют собственный фреймворк ig2vec \cite{ig2vec}, основанный на 
  word2vec, для представления изображений в векторном виде, а дальше, 
  сравнивая вектора пользователей, в онлайн режиме формируют бесконечную
  персонализированную ленту изображений.

  Из открытых инструментов для построения рекомендательных систем, мне
  удалось найти фреймворк \textbf{SurPRISE} \cite{sur} для языка программирования Python, 
  реализованный с помощью SciKits \cite{SciKits} (сокращенно от scipy Toolkits) -- 
  это дополнительные пакеты для SciPy \cite{Scipy}, размещенные и разработанные 
  отдельно и независимо от основного дистрибутива SciPy. SciPy -- 
  это основанная на Python экосистема программного обеспечения с 
  открытым исходным кодом для математики, естественных наук и 
  инженерии.

\subsection{Обзор методов калибровки рекомендательных систем}
\subsubsection{Расстояние Кульбака-Лейблера}
В ходе работы был проведен обзор литературы и найдено два 
метода потенциально подходящих для решения поставленной проблемы. 
Первый метод заключается в пересчете целевой метрики с помощью 
расстояния Кульбака-Лейблера (\ref{eq:KL}) \cite{bib4}.
\begin{equation}
  C_{KL}(p,q) = KL(p||\tilde{q}) = \sum_g p(g|u)\log\frac{p(g|u)}{\tilde{q}(q|u)},
  \label{eq:KL}
\end{equation}
где ${p}$ это целевое распределение жанра ${g}$ для пользователя 
$u$, $q$ -- полученное распределение жанров для пользователя.
Во избежание случая $q(g|u)=0$, будем использовать 
\begin{equation}
  \tilde{q}(q|u) = (1-\alpha) \cdot q(g|u) + \alpha \cdot p(g|u)
\end{equation} с очень маленьким $\alpha>0$, такое что $q\approx\tilde{q}$.

Сама же калибровка выполняется по формуле: 
\begin{equation}
  \label{eq:Calibrated}
  I^*=\arg \max_{I, |I|=N} (1-\lambda) \cdot s(I) - \lambda \cdot C_{KL}(p,q(I)),
\end{equation} где $\lambda \in [0,1]$, которая определяет 
компромисс между расстоянием Кульбака-Лейблера и значением 
метрики полученным рекомендательной системой. $s(I)=\sum_{i\in I}s(i)$, 
где $s(i)$ -- степень уверенности, что фильм $i$ подойдет 
пользователю, предсказанная рекомендательной системой.
\subsubsection{Метод Сент-Лагю}
Второй алгоритм является адаптацией метода Сент-Лагю. \cite{bib5} 
Метод Сент-Лагю, был изобретен французским математиком Андре 
Сент-Лагю для пропорционального распределения мандатов в 
правительстве. Суть метода заключается в поочередном присуждении 
мандатов партии с наибольшей квотой, которая на каждом шаге 
считается по формуле $\frac{V}{2s+1}$, где $V$ -- количество 
голосов, полученных партией, s -- количество мандатов, 
выделенных партии на данном шаге.

Можно модифицировать данный метод под наш случай. 
Имея список наиболее релевантных фильмов, мы будем 
формировать новый список, выбирая фильмы по одному 
методом Сент-Лагю, только вместо партий у нас будут 
жанры, а голоса, полученные партией, заменятся на количество 
понравившихся пользователю фильмов конкретного жанра.

\subsection{Обзор метрик качества методов калибровки рекомендаций}
В качестве метрики качества рекомендательной системы будет 
использоваться точность (\ref{eq:prec}) (precission) --  это доля релевантных экземпляров 
среди извлеченных экземпляров \cite{bib6}, также называемая положительная 
прогностическая ценность и рассчитывается по формуле \begin{equation}
    Precision = \frac{tp}{tp+tn},
    \label{eq:prec}
  \end{equation}
  где ${tp}$ -- true positive, количество элементов корректно принятых алгоритмом, 
  ${tn}$ -- true negative, количество элементов корректно отклоненных алгоримтом. 
  
  Метод оценки схожести распределений предложен в статье от Netflix \cite{bib4}, 
  там используется дивергенция Кульбака-Лейблера (\ref{eq:KL}) из класса ${f}$-дивергенций ${\displaystyle D_{f}(P\parallel Q)}$,  определяющих в общем случае несимметричную меру расхождения между двумя распределениями вероятностей ${P}$ и ${Q}$.

