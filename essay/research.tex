\specialsection{Обзор литературы}
Калибровка рекомендация на данный момент имеет несколько направлений,
 первое заключается в получении более справедливых рекомендаций, а второе 
 в учете всех интересов пользователя при генерации рекомендаций. 
 В данной работе я решил рассмотреть именно второе.

  Второе направление более сосредоточено на диверсификации, 
  полноте и разнообразии рекомендации. Основным инструментом 
  калибровки является переранжирование списка уже сгенерированных
   рекомендаций. В работе \cite{bib3} используется выделение подпрофилей, 
   на основе положительных оценок, 
   для каждого пользователя, которые отражают 
   определенные интересы. Другой метод предлагают исследователи из 
   Netflix \cite{bib4}, он основан на расстоянии Кульбака-Лейблера, 
   которое применяется для определения схожести пользовательского 
   распределения и сгенерированнного. В исследовании \cite{bib5} 
   используется метод Сент-Лагю для выбора N наиболее релевантных и 
   разнообразных рекомендаций. Обычно метод Сент-Лагю применяется для 
   распределения мест в правительстве, но в рассмотренной работе, его 
   модифицировали для рекомендаций фильмов.