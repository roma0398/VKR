\specialsection{Заключение}
\subsection*{Итоги работы}
В рамках работы были выполнены следующие задачи:
\begin{itemize}
    \item Обзор литературы на тему рекомендательных систем и их калибровки;
    \item Обзор существующих методов решения проблемы;
    \item Реализован алгоритм на языке Python и проведены эксперименты на данных;
    \item Проведено сравнение полученных результатов калибровки между собой и с готовой реализацией из фреймворка;
\end{itemize}

Цель работы была достигнута, реализованный алгоритм калибровки рекомендательных систем
заметно улучшает учет пользовательских интересов в рекомендациях и немного повышает точность рекомендаций.
В сравнении с готовым решением, калибровка все еще лучше в плане удовлетворения пользовательских интересов, 
но по точности немного проигрывает.

\subsection*{Практическое применение}
Реализованный алгоритм можно применять практически в любой рекомендательной системе, где фигурируют какие-то классы объектов рекомендаций, а не только для фильмов, как представлено в эксперименте.
Например, в музыкальных сервисах, социальных сетях, новостных агрегаторах, интернет-магазинах.