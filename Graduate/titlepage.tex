% --------------------- Стандарт СПбГУ для ВКР --------------------------
% Автор: Тоскин Николай, itonik@me.com
% Если заметили ошибку, напишите на email
% Если хотите добавить изменение самостоятельно, GitHub: . PR-s welcome!
% Использованы материалы:
% habr.com/ru/post/144648/
% cpsconf.ru
% Текст:
% http://edu.spbu.ru/images/data/normativ_acts/local/20181030_10432_1.pdf
% Титульный лист:
% http://edu.spbu.ru/images/data/normativ_acts/local/20180703_6616_1.pdf
% -----------------------------------------------------------------------

% Титульный лист диплома СПбГУ
% Временное удаление foot на titlepage
\newgeometry{left=30mm, top=20mm, right=15mm, bottom=20mm, nohead, nofoot}
\begin{titlepage}
\begin{center}
% Первый символ съедается, первым знаком поставлен Ы
\text{Санкт--Петербургский государственный университет}\\
\textbf{Кафедра технологии программирования}

\vspace{22mm}

\textbf{\Large Бельков Роман Андреевич} \\[7mm]
% Название
\textbf{\large Выпускная квалификационная работа}\\[12mm]
\textbf{\Large Калибровка рекомендательных систем. Поиск гетерогенного эффекта и нетипичных пользователей для задач рекомендаций контента}

\vspace{12mm}
Направление 01.03.02 \\«Прикладная математика и информатика»\\
Основная образовательная программа СВ.5005.2015
«Прикладная математика, фундаментальная информатика и программирование»\\

\vspace{18mm}

% Научный руководитель, рецензент
% Сходить в уч отдел и узнать, правильно ли
\begin{flushleft}
{\setlength{\leftskip}{22em}
 {Научный руководитель:} \\
 кандидат техн. наук, \\
 доцент \\ Блеканов И. С. \\
}
\end{flushleft}
\begin{flushleft}
    \setlength{\leftskip}{22em}
    {Рецензент:} \\
    старший преподаватель \\ Давыденко А. А.
    \end{flushleft}
\vfill

{Санкт-Петербург}
\par{2020 г.}
\end{center}
\end{titlepage}
\restoregeometry
\addtocounter{page}{1}