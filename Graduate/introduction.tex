
\specialsection{Введение}

В настоящее время, количество информации растет очень быстрыми 
темпами и чтобы предоставлять наиболее релевантную и полезную 
для пользователей информацию, 
в веб сервисах начали использовать рекомендательные системы.  
Рекомендательные системы обеспечивают персонализированный 
пользовательский опыт во многих различных областях применения, 
включая интернет-магазины, социальные сети и потоковое
воспроизведение музыки/видео.  
Рекомендательная система -- это алгоритм, который предсказывает
 наиболее интересные
 объекты для конкретного пользователя на основе некоторой 
 информации о нем.

 Одна из проблем, которые имеют рекомендательные системы -- это 
 удовлетворение не всех интересов пользователя, рассмотрим пример 
 рекомендательной системы фильмов.
 Если пользователь посмотрел, 80 артхаусных фильмов и 20 
 комедий, то вполне разумно ожидать, что персонализированный 
 список рекомендуемых фильмов будет состоять примерно из 80\% 
 артхаусных и 20\% комедий. Это важное свойство, известное 
 как калибровка, недавно получило новое внимание в контексте 
 справедливости машинного обучения. В рекомендуемом списке 
 элементов калибровка гарантирует, что различные области 
 интересов пользователя будут отражены в соответствующих
  пропорциях. 
  
  Калибровка особенно важна в свете того факта,
   что рекомендательные системы, оптимизированные в сторону
    точности
  в обычном автономном режиме, могут легко привести к 
  рекомендациям, где меньшие интересы пользователя 
  вытесняются основными интересами пользователя.
 В этой статье мы покажем, что рекомендательные системы,
  обученные точности, могут легко генерировать списки 
  рекомендуемых элементов, которые фокусируются на основных 
  областях интересов пользователя, в то время как меньшие
   области интересов пользователя, как правило, недопредставлены 
   или даже отсутствуют. 
 Со временем такие несбалансированные рекомендации несут 
 в себе риск постепенного сужения областей интересов 
 пользователя -- что аналогично эффекту пузыря фильтров.
  Эта проблема также применима в случае нескольких пользователей,
   совместно использующих одну учетную запись, когда интересы 
   менее активных пользователей в рамках одной учетной записи 
   могут быть вытеснены в рекомендациях.

 Калибровка -- это общая концепция машинного обучения, 
 и в последнее время она переживает возрождение в 
 контексте справедливости алгоритмов машинного обучения.
  Алгоритм классификации называется калиброванным, если 
  прогнозируемые пропорции различных классов согласуются 
  с фактическими пропорциями точек данных в имеющихся данных. 
