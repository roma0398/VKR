\specialsection{Заключение}
\subsection*{Итоги работы}
В рамках работы были выполнены следующие задачи:
\begin{itemize}
    \item Обзор литературы на тему рекомендательных систем и их калибровки;
    \item Обзор существующих методов решения проблемы;
    \item Изучены и реализованы, на языке Python, два алгоритма калибровки рекомендаций;
    \item Проведены эксперименты на выбранном наборе данных;
    \item Проведено сравнение полученных результатов калибровки между собой и с готовой реализацией из фреймворка;
    \item Построены графики для наглядного сравнения распределения классов интересов пользователей.
\end{itemize}

Цель работы была достигнута, реализованный алгоритм калибровки рекомендательных систем
заметно улучшает учет пользовательских интересов в рекомндациях и немного повышает точность рекомендаций.
В сравнении с готовым решением, калибровка все еще лучше в плане удовлетворения пользовательских интересов, 
но по точности немного проигрывает.

\subsection*{Практическое применение}
Реализованный алгорит можно применять практически в любой рекомендательной системе, где фигурируют какие-то классы объктов рекомендаций, а не только для фильмов, как представлено в эксперименте.
Например в музыкальных сервисах можно заменить жанры фильмов на жанры музыки и использовать тот же метод.
В социальных сетях вместо жанров можно использовать теги записей, направленность группп и каналов. В новостных агрегаторах ориентироваться на тематику новости.
В интернет-магазинах классами будут являться категории товаров.

\subsection*{Дальнейшее развитие}
В дальнейшем для улучшение алгоритма, можно заменить метод применяемы рекомендательной системой на более точный и калибровать его результаты,
возможно точность в этом случает будет даже выше, чем в готовом фреймворке. Также, можно попробовать реализовать идею применения нескольких методов калибровки последовательно.
