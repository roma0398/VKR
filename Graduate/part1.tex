\section{Обзор литературы}
\subsection{Расстояние Кульбака-Лейблера}
В ходе работы был проведен обзор литературы и найдено два метода потенциально подходящих для решения поставленной проблемы. Первый метод заключается в пересчете целевой метрики с помощью расстояния Кульбака-Лейблера (\ref{eq:KL}). \cite{voc1}
\begin{equation}
  C_{KL}(p,q) = KL(p||\tilde{q}) = \sum_g p(g|u)\log\frac{p(g|u)}{\tilde{q}(q|u)},
  \label{eq:KL}
\end{equation}
где ${p}$ это целевое распределение жанра ${g}$ для пользователя $u$, $q$ -- полученное распределение жанров для пользователя.
Во избежание случая $q(g|u)=0$, будем использовать 
\begin{equation}
  \tilde{q}(q|u) = (1-\alpha) \cdot q(g|u) + \alpha \cdot p(g|u)
\end{equation} с очень маленьким $\alpha>0$, такое что $q\approx\tilde{q}$.

Сама же калибровка выполняется по формуле: 
\begin{equation}
  \label{eq:Calibrated}
  I^*=\arg \max_{I, |I|=N} (1-\lambda) \cdot s(I) - \lambda \cdot C_{KL}(p,q(I)),
\end{equation} где $\lambda \in [0,1]$, которая определяет компромисс между расстоянием Кульбака-Лейблера и значением метрики полученным рекомендательной системой. $s(I)=\sum_{i\in I}s(i)$, где $s(i)$ -- степень уверенности, что фильм $i$ подойдет пользователю, предсказанная рекомендательной системой.
\subsection{Метод Сент-Лагю}
Второй алгоритм является адаптацией метода Сент-Лагю. \cite{voc2} Метод Сент-Лагю, был изобретен французским математиком Андре Сент-Лагю для пропорционального распределения мандатов в правительстве. Суть метода заключается в поочередном присуждении мандатов партии с наибольшей квотой, которая на кажом шаге считается по формуле $\frac{V}{2s+1}$, где $V$ -- количесвто голосов, полученных партией, s -- количество мандатов, выделенных партии на данном шаге.

Можно модифицировать данный метод под наш случай. Имея список наиболее релевантных фильмов, мы будем формировать новый список, выбирая фильмы по одному методом Сент-Лагю, только вместо партий у нас будут жанры, а голоса, полученные партией, заменятся на количество понравившихся пользователю фильмов конкретного жанра.
