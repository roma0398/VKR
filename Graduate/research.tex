\specialsection{Обзор литературы}
Калибровка рекомендация на данный момент имеет несколько направлений,
 первое заключается в получении более справедливых рекомендаций, а второе 
 в учете всех интересов пользователя при генерации рекомендаций. 
 
 Рассмотрим первое направление. Как правило, в рекомендательных 
 системах речь идет о рекомендациях товаров пользователям. 
 Однако в контексте справедливости имеет смысл абстрагироваться
  от объектов и субъектов. Например, при рекомендациях предложений о 
  работе испытуемыми являются люди, которые могут подвергаться 
  дискриминации по признаку расы или пола; при поиске поездки 
  рекомендуемыми объектами являются таксисты; в социальных сетях 
  предложения друзей включают отдельных людей в качестве субъектов 
  и объектов. Таким образом, справедливость в рекомендателями 
  могут иметь несколько точек зрения, как описано в \cite{bib1}: 
  справедливость для субъектов, называются потребительской справедливостью или 
  C-справедливостью, а справедливость для объектов, называется 
  справедливостью производителя или P-справедливостью. 

  Дискриминация в рекомедательных системах является отдельной задачей,
  которая достаточно хорошо описана в статье об исследовании честности
  иснтрументов для прогнозирования рецидива преступлений \cite{bib2}. 

  Второе направление более сосредоточено на диверсификации, 
  полноте и разнообразии рекомендации. Основным интсрументом 
  калибровки является переранжирование списка уже сгенерированных
   рекомендаций. В работе \cite{bib3} используется выделение подпрофилей, 
   на основе положительных оценок, 
   для каждого пользователя, которые отражают 
   определенные интересы. Другой метод предлагают исследователи из 
   Netflix \cite{bib4}, он основан на расстоянии Кульбака-Лейблера, 
   которое применяется для опеределения схожести пользовательского 
   распределения и сгенерированнного. В исследовании \cite{bib5} 
   используется метод Сент-Лагю для выбора N наиболее релевантных и 
   разнообразных рекомендаций. Обычно метод Сент-Лагю применяется для 
   распределения мест в правительстве, но в рассмотренной работе, его 
   модифицировали для рекомендаций фильмов.